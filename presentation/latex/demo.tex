%%%%%%%%%%%%%%%%%%%%%%%%%%%%%%%%%%%%%%%%%%%%%%%%%%%%%%%%%%%%%%%%%%%%%%%%%%%%%%%
%%%%%%%%%%%%%%%%%%%%%%%%%%%%%%%%%%%%%%%%%%%%%%%%%%%%%%%%%%%%%%%%%%%%%%%%%%%%%%%

\documentclass{beamer}

% Disable annoying warning about unavailable font size.
% Computer Modern does not have the font size beamer wants.
\let\Tiny=\tiny

%%%%%%%%%%%%%%%%
% Beamer theme %
%%%%%%%%%%%%%%%%

% Fira Fonts may not work out of the box on Windows.
% Use classic 'Computer Modern' fonts instead by disabling firafonts with the
% 'nofirafonts' switch.

% Decide on the look of the footline. It can be 
% progressing from left to right with each slide (standard)
% or be suppressed (option "none") or show the short title
% and short author as specified below in the \title and 
% \author command (option "fullbar").

% \usetheme{bse}
\usetheme[numbering=fullbar]{bse}
% \usetheme[nofirafonts]{bse}
% \usetheme[numbering=none, nofirafonts]{bse}
% \usetheme[numbering=fullbar, nofirafonts]{bse}


% You can specify custom colours.
% See beamerthemebse.sty for details.
% \definecolor{main}{RGB}{64, 64, 64}
% \definecolor{background}{RGB}{239, 239, 239}

%%%%%%%%%%
% Tables %
%%%%%%%%%%

% Better table ruler separation
\usepackage{booktabs} 

%%%%%%%%%%%%%%
% Typography %
%%%%%%%%%%%%%%

% Conventions for date, time, hyphenation etc.
\usepackage[UKenglish]{babel}

%encoding
\usepackage[utf8]{inputenc}
\usepackage[T1]{fontenc}

%%%%%%%%%%%%%%%%
% Bibliography %
%%%%%%%%%%%%%%%%

% Use natbib for author-year style
\usepackage[round]{natbib}
\bibliographystyle{plainnat}

%%%%%%%%%%%%%%
% Title page %
%%%%%%%%%%%%%%

\title[Short Title]{The Berlin School of Economics}
\subtitle{A Beamer Template}
\author[Humboldt \& Humboldt]{W. Humboldt\texorpdfstring{\\}{,} A. Humboldt}
\institute{Berlin School of Economics\\ \footnotesize Conference XYZ}
\date{\today}

\thanksmessage{Or use the ``thank you'' slide.}

%%%%%%%%%%%%%%%%%%%%%%%%%%%%%%%%%%%%%%%%%%%%%%%%%%%%%%%%%%%%%%%%%%%%%%%%%%%%%%%
%%%%%%%%%%%%%%%%%%%%%%%%%%%%%%%%%%%%%%%%%%%%%%%%%%%%%%%%%%%%%%%%%%%%%%%%%%%%%%%

\begin{document}

    \begin{frame}
        \maketitle
    \end{frame}
    
    \begin{frame}[t]\frametitle{Agenda}
        \tableofcontents
    \end{frame}

    % Use starred version (e.g. \section*{Section name})
    % to disable (sub)section page.
    \section{Introduction}
    \subsection{A First View}
    \begin{frame}{Simple frame}
        This is a simple frame.
    \end{frame}

    \begin{frame}[plain]{Plain frame}
        This is a frame with plain style and it is numbered.
    \end{frame}
    
    \subsection{A Second View}
    \begin{frame}[t]
        This frame has an empty title and is aligned to top.
    \end{frame}
    
    \begin{frame}[noframenumbering]{No frame numbering with a really long title on the frame, which still formats nicely. Does it?}
        This frame is not numbered and is citing \cite{knuth74}.
    \end{frame}
    
    \begin{frame}{Typesetting and Math}
        The packages \texttt{inputenc} and \texttt{FiraSans}\footnote{\url{https://fonts.google.com/specimen/Fira+Sans}}\textsuperscript{,}\footnote{\url{http://mozilla.github.io/Fira/}} are used to properly set the main fonts. Unless you are on Windows and don't want to install \texttt{FiraSans}. Then you can use old school \texttt{Computer Modern}.
        \vfill
        This theme provides styling commands to typeset \emph{emphasized}, \alert{alert}, \textbf{bold}, \textcolor{BSElightgreen}{example text}, \dots
        \vfill
        \texttt{FiraSans} and \texttt{Computer Modern} also provides support for mathematical symbols:
        \begin{equation*}
            e^{i\pi} + 1 = 0.
        \end{equation*}
    \end{frame}

    \section{Details}
    \begin{frame}{Blocks}
        \begin{block}{Block}
            Text.
        \end{block}
        \pause
        \begin{alertblock}{Alert block}
            Alert \alert{text}.
        \end{alertblock}
        \pause
        \begin{exampleblock}{Example block}
            Example \textcolor{BSElightgreen}{text}.
        \end{exampleblock}
    \end{frame}
    
    \begin{frame}{Lists}
        \begin{columns}[t, onlytextwidth]
            \column{0.33\textwidth}
                Items:
                \begin{itemize}
                    \item Item 1
                    \begin{itemize}
                        \item Subitem 1.1
                        \item Subitem 1.2
                    \end{itemize}
                    \item Item 2
                    \item Item 3
                \end{itemize}
            
            \column{0.33\textwidth}
                Enumerations:
                \begin{enumerate}
                    \item First
                    \item Second
                    \begin{enumerate}
                        \item Sub-first
                        \item Sub-second
                    \end{enumerate}
                    \item Third
                \end{enumerate}
            
            \column{0.33\textwidth}
                Descriptions:
                \begin{description}
                    \item[First] Yes.
                    \item[Second] No.
                \end{description}
        \end{columns}
    \end{frame}
\setbeamertemplate{caption}[numbered]
    \begin{frame}{Figures and Tables}
        \begin{columns}
            \column{0.4\textwidth}
                \begin{figure}
                    \centering
                    \includegraphics[width=\linewidth]{images/logos/bse_round.png}
                    \caption{Figure caption.}
                    \label{fig:bse_logo}
                \end{figure}
                
            \column{0.6\textwidth}
                \begin{table}
                    \centering
                    \begin{tabular}{rcc}
                    	\toprule
                         & Heading 1 & Heading 2 \\
                        \toprule
                        Row 1 & \(v_{11}\) & \(v_{12}\) \\
                        Row 2 & \(v_{21}\) & \(v_{22}\) \\
                        Row 3 & \(v_{31}\) & \(v_{32}\) \\
                        \bottomrule
                    \end{tabular}
                    \caption{Table caption.}
                    \label{tab:demo}
                \end{table}
        \end{columns}
    \end{frame}
    
    \begin{frame}[focus]
        \vskip 1em
        Thanks for using the \textbf{BSE} theme!\\[-5mm]
        \hrulefill\\
        \vskip 1em
        \usebeamercolor[fg]{BSEwhite}
        \usebeamerfont{footline}
        \textbf{Berlin School of Economics}\\
        Editor: \textit{Name}\\
        DIW Graduate Center\\
        Mohrenstraße 58, 10117 Berlin\\
        \url{www.diw.de/gc}\\
        \href{mailto:example@berlin-econ.de}{example@berlin-econ.de}
        \vskip 1em
        \footnotesize Include a license, if you want.
        \makelicense
	\vskip 1.5em
        \vfill
        \begin{figure}[b]
        	\setlength{\fboxsep}{0pt}%
        	\setlength{\fboxrule}{0pt}
         \fbox{\hspace{-0.83cm}\includegraphics[width=13cm]{images/logos/bse_members.png}}
    	\end{figure}
    	\vfill{}
    \end{frame}
 
    \makethanks
    \appendix
    \begin{frame}{References}
        \nocite{*}
        \bibliography{demo_bibliography}
    \end{frame}
    
    \begin{frame}{Backup frame}
        \usebeamercolor[fg]{normal text}
        This is a backup frame, useful to include additional material for questions from the audience.
        \vfill
        The package \texttt{appendixnumberbeamer} is used not to number appendix frames.
    \end{frame}
\end{document}

%%%%%%%%%%%%%%%%%%%%%%%%%%%%%%%%%%%%%%%%%%%%%%%%%%%%%%%%%%%%%%%%%%%%%%%%%%%%%%%
%%%%%%%%%%%%%%%%%%%%%%%%%%%%%%%%%%%%%%%%%%%%%%%%%%%%%%%%%%%%%%%%%%%%%%%%%%%%%%%

